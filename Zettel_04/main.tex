\documentclass{article}
\usepackage{fancyhdr}
\usepackage{tabularx}
\usepackage{geometry}
\usepackage{lipsum}
\usepackage{amssymb}
\usepackage{venndiagram}
\usepackage{subcaption}
\usepackage{wrapfig}
\usepackage{multicol}
\usepackage{float}
\usepackage{amsthm}




% Seitenränder einstellen
\geometry{a4paper, left=2.5cm, right=2.5cm, top=2.5cm, bottom=2.5cm}

% Header definieren
\pagestyle{fancy}
\fancyhf{}
\renewcommand{\headrulewidth}{1pt}
\lhead{Anton Jagow, Lukas Dzielski}
\rhead{Ruprecht-Karls-Universität Heidelberg \\ MaFIn \\ Max-Emanuel Hlawatsch}
\lfoot{}
\rfoot{\thepage}

\begin{document}

\section*{Übungszettel \#4} % Setze die Nummer des Zettels

\begin{center}
    \begin{tabular}{|c|c|}
        \hline
        Aufgabe & Punkte \\
        \hline
        1 & \\
        2 & \\
        3 & \\
        \hline
        Gesamt & \\
        \hline
    \end{tabular}
\end{center}

\subsection*{Aufgabe 1}
    \begin{itemize}
        \item [(a)] \(f_1(\{3, 4, 5, 6\}) = \{9, 16, 25, 36\}\)
        \item [(b)] \(f_2(\{3, 4, 5, 6\}) = \{4, 9\}\)
        \item [(c)] \(f_3(\{30, \cdots, 50\}) = \{6, 7\}\)
        \item [(d)] \(f_4(\{30, \cdots, 50\}) = \{26, \cdots, 49\}\)
        \item [(e)] \begin{itemize}
            \item [(\(f_1\))] \textbf{injektiv}, da die Wertemenge schneller wächst als Definitionsmenge. Deswegen gibt es für jedes x aus der Wertemenge auf jeden fall nur ein y aus Definitionsmenge.\\ \textbf{Nicht surjektiv}, da es y gibt s.d. kein x zugeordnet ist. \(\Rightarrow y = 2 \) 
            \item [(\(f_2\))] \textbf{Nicht injektiv}, man wähle \(x_1 = 2, x_2 = 3\) es gilt das \(x_1 \not = x_2\) aber \(f(2) = 4 = f(3)\) \\ \textbf{Nicht surjektiv}, da es y gibt s.d. kein x zugeordnet ist. \(\Rightarrow y = 2 \)
            \item [(\(f_3\))] \textbf{injektiv} und \textbf{surjektiv} da die Funktion genau alle werte aus Q trifft. 
            \item [(\(f_4\))] \textbf{Nicht injectiv}, man wähle \(x_1 = 2, x_2 = 3\) es gilt das \(x_1 \not = x_2\) aber \(f(2) = 4 = f(3)\) \\ \textbf{surjektiv} im gegensatz zu \((f_2)\) identisch aber die menge hat nur quadratzahlen also gibt es kein y was nicht einem x zugeordnet ist.
        \end{itemize}
    \end{itemize}

\subsection*{Aufgabe 2}
    \begin{itemize}
        \item [(a)] \begin{itemize}
            \item [(i)] \(A_0 = \{2\}\) \\ \(\Rightarrow q^{-1}(q(2)) = q^{-1}(4) = \{2, 3, 4\}\) \\ \(\{2\} \not = \{2, 3, 4\}\) \\
            \item [(ii)] \(A'_0 = \{1\}\) \\ \(\Rightarrow q^{-1}(q(1)) = q^{-1}(1) = 1\) \\ \(\{1\} = \{1\}\) \\
            \item [(iii)] \(B_0 = ?\) \\ \( \Rightarrow q(q^{-1}(B_0))\) \\
            \item [(iv)] \(B'_0 = \{4\}\) \\ \(\Rightarrow q(q^{-1}(4)) = q(\{2, 3, 4\}) = 4\) \\

        \end{itemize}
        \item [(b)]
    \end{itemize}
    
\subsection*{Aufgabe 3}

\end{document}
