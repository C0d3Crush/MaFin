\documentclass{article}
\usepackage{fancyhdr}
\usepackage{tabularx}
\usepackage{geometry}
\usepackage{lipsum}
\usepackage{amssymb}
\usepackage{venndiagram}
\usepackage{subcaption}
\usepackage{wrapfig}
\usepackage{multicol}
\usepackage{float}
\usepackage{amsthm}




% Seitenränder einstellen
\geometry{a4paper, left=2.5cm, right=2.5cm, top=2.5cm, bottom=2.5cm}

% Header definieren
\pagestyle{fancy}
\fancyhf{}
\renewcommand{\headrulewidth}{1pt}
\lhead{Anton Jagow, Lukas Dzielski}
\rhead{Ruprecht-Karls-Universität Heidelberg \\ MaFIn \\ Max-Emanuel Hlawatsch}
\lfoot{}
\rfoot{\thepage}

\begin{document}

\section*{Übungszettel \#3} % Setze die Nummer des Zettels

\begin{center}
    \begin{tabular}{|c|c|}
        \hline
        Aufgabe & Punkte \\
        \hline
        1 & \\
        2 & \\
        3 & \\
        4 & \\
        % Fügen Sie hier weitere Aufgaben hinzu
        \hline
        Gesamt & \\
        \hline
    \end{tabular}
\end{center}

\subsection*{Aufgabe 1}
    Die Relation R über A ist Reflexiv, Transitiv sowie Symetrisch dann ist diese nach definition eine Äquivalenzklasse. Wegen Symetrie gilt \( \forall x, y \in A \ xRy\) und \(yRx\). Wenn nun \(\forall x, y \in A \quad x = y\) gilt ist die Regel für Antisymetrie nicht verletzt. So kann eine soche Relation existieren. 

\subsection*{Aufgabe 2}
    \begin{center}
        \begin{tabular}{|c|c|c|c|c|}
            \hline
             & \(\leq_a\) & \(\leq_b\) & \(\leq_c\) & \(\leq_d\) \\
            \hline
            reflexiv & \(\checkmark\) & \(\checkmark\) & \(\checkmark\) & \(\checkmark\) \\
            \hline
            transitiv & \(\checkmark\) & \(\checkmark\) & \(\checkmark\) & \(\checkmark\) \\
            \hline
            antisymetrie & \(\checkmark\) & \(\checkmark\) & \(\times\) & \(\times\) \\
            \hline
            konnex & \(\checkmark\) & \(\times\) & \(\checkmark\) & \(\times\) \\
            \hline
             & Ordnung & partielle Ordnung & Präordnung & partielle Präordnung \\
            \hline
        \end{tabular}        
    \end{center}
    
\subsection*{Aufgabe 3}
\begin{itemize}
    \item [(a)] \begin{proof}
        Zu zeigen: Die Erreichbarkeitsrelation \(\curvearrowright\) ist sowohl transitiv als auch reflexiv.\\
        \begin{itemize}
            \item [reflexivität:] In der Aufgabestellung steht: "Nach Definition ist jeder Knoten von sich selbst aus erreichbar, \(v \curvearrowright v\) gilt für alle \(v\) in V "
            \item [transitivität:] \(v_1, \cdots, v_t\) und \(v_t, \cdots, v_j\) sind zwei folgen von Kanten wobei die Paare \((v_i, v_i+1)\) im Graphen existieren. Da aber jetzt \(v_j\) aus \(v_1\) erreichbar ist da es für jeden übergang ein Paar gibt können wir auch schreiben: \(v_1, \cdots, v_j\). Also gilt: 
            \[v_1 \curvearrowright v_t \wedge v_t \curvearrowright v_j \Leftrightarrow v_1 \curvearrowright v_j\] 
        \end{itemize}
    \end{proof}
    \item [(b)] G ist in jedem Knoten aus zusammenhöngend. Zwei beliebige Knoten \(v\) und \(w\) aus V haben einen gerichteten Weg von v
    \(v\) nach \(w\) und \(w\) nach \(v\)
\end{itemize}

\subsection*{Aufgabe 4}
    \begin{itemize}
        \item [(a)] \begin{proof}
            Zu zeigen: Die Erreichbarkeitsrelation \(\curvearrowright\) ist eine Äquivalenzrelation auf der \\Knotenmenge V. \\Also muss sie reflexiv, transitiv und symetrisch sein. Es muss nur Symetrie gezeigt werden da Aufgabe 3 analog dazu ist.
            \begin{itemize}
                \item [symetrie:] Da ungerichtet, gilt \(v_i \curvearrowright v_{i+1}\) und \(v_{i+1} \curvearrowright v_{i}\)
            \end{itemize}
            \[\Rightarrow [v_i]_E = \{v_{i+1} \in V : v_i \curvearrowright v_{i+1} \wedge v_{i+1} \curvearrowright v_{i} \}\]
        \end{proof}
        \item[(b)] Es gibt zu je 2 beliebigen Knoten \(v\) und \(w\) in V einen ungerichteten Weg in G mit v als Startknoten und \(w\) als Endknoten.
    \end{itemize}

\end{document}
