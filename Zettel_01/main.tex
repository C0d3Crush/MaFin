\documentclass{article}
\usepackage{fancyhdr}
\usepackage{tabularx}
\usepackage{geometry}
\usepackage{lipsum}
\usepackage{amssymb}
\usepackage{venndiagram}
\usepackage{subcaption}
\usepackage{wrapfig}
\usepackage{multicol}
\usepackage{float}




% Seitenränder einstellen
\geometry{a4paper, left=2.5cm, right=2.5cm, top=2.5cm, bottom=2.5cm}

% Header definieren
\pagestyle{fancy}
\fancyhf{}
\renewcommand{\headrulewidth}{1pt}
\lhead{Anton Jagow, Lukas Dzielski}
\rhead{Ruprecht-Karls-Universität Heidelberg \\ MaFIn \\ Max-Emanuel Hlawatsch}
\lfoot{}
\rfoot{\thepage}

\begin{document}

\section*{Übungszettel \#1} % Setze die Nummer des Zettels

\begin{center}
    \begin{tabular}{|c|c|}
        \hline
        Aufgabe & Punkte \\
        \hline
        1 & \\
        2 & \\
        3 & \\
        4 & \\
        % Fügen Sie hier weitere Aufgaben hinzu
        \hline
        Gesamt & \\
        \hline
    \end{tabular}
\end{center}

\section*{Aufgabe 1}

Die Falsche Annahme ist das es sich um ein Dreieck handelt. Die zwei kleineren Dreiecke die man verwendet um das Objekt zusmmenzusetzen haben nicht den gleichen Winkel daher erscheint es als würde man mehr oder weniger Fläche haben. 

\section*{Aufgabe 2}
    \subsection*{(i)}
        \[A_{11} \cup A_{999}\] 
        \[= A_{999} \cap A_{4}\] 
        \[=A_{4}\]
    \subsection*{(ii)}
        \[(A_7 / A_5) \times A_2\]
        \[=\{6, 7\} \times \{1, 2\}\]
        \[=\{6, 1\}, \{6, 2\}, \{7, 1\}, \{7, 2\}\]
    \subsection*{(iii)}
        \[Pot(A_{11} / A_{8})\]
        \[= Pot(\{9, 10, 11\})\]
        \[=\{\varnothing, \{9\}, \{10\}, \{11\}, \{9, 10\}, \{9,11\}, \{10, 11\}, \{9, 10, 11\}\} \]
    \subsection*{(iv)}
        \[Pot(A_2) \vartriangle Pot(A_3)\]
        \[= \{\varnothing, \{1\}, \{2\}, \{1,2\}\}\vartriangle  \{\varnothing, \{1\}, \{2\}, \{1,2\}, \{1, 3\}, \{2,3\}, \{1, 2, 3\}\}\]
        \[=\varnothing \cup \{\{1, 3\}, \{2, 3\}, \{1, 2, 3 \}, \}\]

\section*{Aufgabe 3}
    \begin{center}
        \begin{tabular}{|c|c|}
        \hline
        $i$ & $g(i)$ \\
        \hline
        1 & $g(8)$ \\
        \hline
        2 & $g(7)$ \\
        \hline
        3 & $g(6)$ \\
        \hline
        4 & $g(5)$ \\
        \hline
        5 & $g(4)$ \\
        \hline
        6 & $g(3)$ \\
        \hline
        7 & $g(2)$ \\
        \hline
        8 & $g(1)$ \\
        \hline
        \end{tabular}
    \end{center}

\section*{Aufgabe 4}
\subsection*{a)}
\begin{multicols}{2} % Teile den Text in zwei Spalten auf

    
    \begin{figure}[H] % Erstes Venn-Diagramm
        \centering
        \begin{venndiagram2sets}
            \fillACapB
        \end{venndiagram2sets}
        \caption{\(A \cap B\)}
    \end{figure}
    
    
    \columnbreak % Wechsel zur zweiten Spalte
    
    \begin{figure}[H] % Zweites Venn-Diagramm
        \centering
        \begin{venndiagram2sets}
            \fillA \fillB
        \end{venndiagram2sets}
        \caption{\(A \cup B\)}
    \end{figure}
    
    
    \end{multicols}
    
\subsection*{b)}
    \begin{center}
        \begin{tabular}{|c|c|c|c|c|c|}
        \hline
        A & B & \(A \cup B\) & \(A \cap B\) & \(\neg A \cap \neg B\) & \(\neg (\neg A \cap \neg B)\) \\
        \hline
        0 & 0 & 0 & 0 & 1 & 0 \\
        \hline
        0 & 1 & 1 & 0 & 0 & 1 \\
        \hline
        1 & 0 & 1 & 0 & 0 & 1 \\
        \hline
        1 & 1 & 1 & 1 & 0 & 1 \\
        \hline
        \end{tabular}
    \end{center}
    Man erkennt hier das \(A \cup B\) und \(\neg (\neg A \cap \neg B)\) gleich sind für alle möglichen eingabewerte. Also ist gezeigt \(A \cup B = \neg (\neg A \cap \neg B)\)
    
\subsection*{c)}
\end{document}
