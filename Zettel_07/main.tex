\documentclass{article}
\usepackage{fancyhdr}
\usepackage{tabularx}
\usepackage{geometry}
\usepackage{lipsum}
\usepackage{amssymb}
\usepackage{venndiagram}
\usepackage{subcaption}
\usepackage{wrapfig}
\usepackage{multicol}
\usepackage{float}
\usepackage{amsthm}

\usepackage{caption}





% Seitenränder einstellen
\geometry{a4paper, left=2.5cm, right=2.5cm, top=2.5cm, bottom=2.5cm}

% Header definieren
\pagestyle{fancy}
\fancyhf{}
\renewcommand{\headrulewidth}{1pt}
\lhead{Anton Jagow, Lukas Dzielski}
\rhead{Ruprecht-Karls-Universität Heidelberg \\ MaFIn \\ Max-Emanuel Hlawatsch}
\lfoot{}
\rfoot{\thepage}

\begin{document}

\section*{Übungszettel \#4} % Setze die Nummer des Zettels

\begin{center}
    \begin{tabular}{|c|c|}
        \hline
        Aufgabe & Punkte \\
        \hline
        1 & \\
        2 & \\
        3 & \\
        \hline
        Gesamt & \\
        \hline
    \end{tabular}
\end{center}

\subsection*{Aufgabe 1}
\paragraph*{(i)} Verknüpfungstabelle, addition, d = 6\\
\begin{tabular}{c|ccccccc}
    + & 0 & 1 & 2 & 3 & 4 & 5 & 6 \\
    \hline
    0 & 0 & 1 & 2 & 3 & 4 & 5 & 0 \\
    1 & 1 & 2 & 3 & 4 & 5 & 0 & 1 \\
    2 & 2 & 3 & 4 & 5 & 0 & 1 & 2 \\
    3 & 3 & 4 & 5 & 0 & 1 & 2 & 3 \\
    4 & 4 & 5 & 0 & 1 & 2 & 3 & 4 \\
    5 & 5 & 0 & 1 & 2 & 3 & 4 & 5 \\
    6 & 0 & 1 & 2 & 3 & 4 & 5 & 0 \\
\end{tabular}
\paragraph*{(ii)} Verknüpfungstabelle, multiplikation, d = 6\\
\begin{tabular}{c|ccccccc}
    * & 0 & 1 & 2 & 3 & 4 & 5 & 6 \\
    \hline
    0 & 0 & 0 & 0 & 0 & 0 & 0 & 0 \\
    1 & 0 & 1 & 2 & 3 & 4 & 5 & 0 \\
    2 & 0 & 2 & 4 & 0 & 2 & 4 & 0 \\
    3 & 0 & 3 & 0 & 3 & 0 & 3 & 0 \\
    4 & 0 & 4 & 2 & 0 & 4 & 2 & 0 \\
    5 & 0 & 5 & 4 & 3 & 2 & 1 & 0 \\
    6 & 0 & 0 & 0 & 0 & 0 & 0 & 0 \\
\end{tabular}
\end{document}
